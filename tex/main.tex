\documentclass[9pt,twocolumn,twoside]{extarticle}

% Packages for bioRxiv format
\usepackage[utf8]{inputenc}
\usepackage[T1]{fontenc}
\usepackage{amsmath,amsfonts,amssymb}
\usepackage{graphicx}
\usepackage{booktabs}
\usepackage{hyperref}
\usepackage{natbib}
\usepackage{geometry}
\usepackage{lineno}
\usepackage{setspace}
\usepackage{authblk}
\usepackage{xcolor}
\usepackage{url}

% bioRxiv page setup
\geometry{
    letterpaper,
    top=0.75in,
    bottom=0.75in,
    left=0.5in,
    right=0.5in,
    columnsep=0.25in
}
\singlespacing
\linenumbers
\modulolinenumbers[5]

% bioRxiv header
\usepackage{fancyhdr}
\pagestyle{fancy}
\fancyhf{}
\fancyhead[L]{\small bioRxiv preprint}
\fancyhead[R]{\small \today}
\fancyfoot[C]{\thepage}
\renewcommand{\headrulewidth}{0.4pt}

% Title and authors for bioRxiv
\title{\textbf{OuterSpace: A tool for extracting and quantifying unique molecular indices}}
% Open to debate

% Authors using authblk package for bioRxiv format
\author[1,2]{Shirley Barrera}
\author[1,2]{DV Klopfenstein}  
\author[1,2]{Rachel Berman}
\author[1,2]{Will Dampier}
% Order open to debate

\affil[1]{Department of Microbiology and Immunology, Drexel University College of Medicine, Philadelphia, PA, United States}
\affil[2]{Center for Molecular Virology and Gene Therapy, Institute for Molecular Medicine and Infectious Disease, Drexel University College of Medicine, Philadelphia, PA, United States}
\affil[*]{Corresponding author}

\date{}

\begin{document}

\maketitle

\begin{abstract}

During modern next generation sequencing it is often important to distinguish between PCR duplicates of the same original molecule.
Many modern sequencing techniques accomplish this by combining pools of oligonucleotides in which an intended region is a unique identifier in each molecule.
When sequenced, these unique molecular indices disambiguate between PCR duplicates and unique copies of the same molecule.
Despite the explosion in usage of these technqiues, there is a lack of a robust toolkit for extracting, analyzing, and visualizing these identifiers.
Outerspace seeks to fill this void.
Using fuzzy-regular expressions the user can easily extract arbitrarily complicated indexing strategies.
Robust error correction techniques are employed to ensure accurate counts in the presence of sequencing noise.
Analysis of large collections can be spread across multiple processes or computers through a Snakemake interface.
Outerspace employs configuration files in order to improve consistency across experiments and the sharing of protocols.
Together, these features make outerspace a useful addition to a bioinformaticians toolbox when handling unique molecular identifiers.

\textbf{Keywords:} bioinformatics, unique molecular identifiers
\end{abstract}

\section{Introduction}

Distinguishing between individual molecules in a population is an important aspect of many modern sequence based technologies.
Early computational strategies for marking duplicate molecules relied on the random shearing nature of tagmenetation enzymes and marked fragments with identical mapping locations as duplications \cite{li2011statistical}. 
However, this technique cannot be employed amplicon based sequencing or modern protocols which intentionally produce fragments with identical ends.
Molecular techniques to mark duplicates utilize pools of nearly identical, but in fact unique, molecules during PCR or ligation steps to affix each original molecule with one instance of the pool.

This is accomplished by incorperating regions of degeneracy nested between constant regions in synthetic oligonucelotides ie. ATGCNNNAGCTA.
When produced, this pattern will be a pool of 64 different oligonucelotides each starting with ATGC and ending with AGCTA.
The middle three nucleotides will be one of A, C, G, or T.
When used during PCR or ligation, each molecule will be 'appended' with one instance of the pool allowing for an ideallized indexing of 64 different molecules.
During sequencing, this unique sequence is read and used for downstream processing.

This strategy has been incorperated into techniques ranging from HIV quantification, CRISPR sreens, lineage tracking, and single-cell transcriptomics.
To quantify HIV genetic variability, PrimerID, utilizes a degenerate cDNA primer to during RT-PCR of viral RNA to uniquely tag each viral genome \cite{jabara2011primerid, zhou2015primerid}.
The single-cell transcriptomics technology employed by 10X Genomics involves performing cDNA synthesis in single-cell emulsions in which each bubble contains a different cell-specific UMI \cite{zheng2017massively}.
The popular CRISPR off-target detection technique GUIDE-seq employs UMIs embeded into y-adapters that are ligated to genomic fragments after shearing \cite{liang2022guidetag}.
% Any other examples?

Transduction of lentiviral libraries with degenerate regions has enabled in vivo lineage tracking and high-scale screening.
So called, barcoded viruses, can be introduced and sampled at different timepoints or from different anatomical regions.
This has been leveraged to quantify viral rebound in SIV and HIV mouse models \cite{fennessey2017genetically, keele2024early, marsden2020tracking}.
When the degenerate region is a functional element such as a CRISPR protospacer or siRNA region each lentivirus causes a unique perturbation. 
In a CRISPR knock out screen (CRISPR-KO) each protospacer targets the coding region of a gene.
Park et al leveraged this strategy to find human genes which impacted HIV replication \cite{park2017genome}. 

Due to proprietary considerations and experimental necessicities, there is no standard construction strategy for UMIs.
This requires an indivudalized approach to extraction, correction, and analysis of UMIs.
Previous computational tools like:
Je, an early java tool focused on Illumina data allowed for the extraction and deduplication of reads \cite{girardot2016je} but is no longer in active development.
AmpUMI can extract single UMIs from unaligned sequences and only export a single instance of each detected UMI \cite{clement2018ampumi}.
UMI-tools, a python toolkit for extracting, correcting, and de-duplicating reads based on UMIs has become a popular pipeline component \cite{smith2017umitools}.
Other tools like CellRanger, handle UMIs for the specific chemistry used by 10X Genomics.
However, all of these tools were designed for Illumina based technology and assume constant sized UMIs.

Long read sequencing introduces additional complications not addressed by previous tools.
Nanopore sequencing is more error-prone and typically introduce insertion or deletion mutations [REF].
This frustrates the matching systems of previous tools which expect an error-profile dominated by rare mismatch errors.
When performing selection-based CRISPR screens or lineage tracing, it is important to have tools for ranking UMIs based on over-representation or tracking them across samples.

To this end we have developed outerspace, a command-line python toolset for extracting, correcting, ranking, and tracking UMIs in long and short-read data.
It employs fuzzy-regular expressions to define patterns allowing for mismatches, insertions, and deletions.
It incorperates the error-correction model from UMI-tools as well size-invariant strategies such as CD-HIT \cite{li2006cdhit}. % CD-HIT engine still needs implementing
Furthermore, it allows the user to develop a reusable and sharable configuration file that define the patterns and all relevant command-line arguments.
It also employs Snakemake to process large experiments in parrallel across cluster systems.
Together, these features make outerspace a useful tool when analyzing UMIs.

\section{Methods}

\subsection{Algorithm Design}

Describe the core algorithms and methodologies used in OuterSpace.

\subsection{Implementation}

Detail the software implementation, including programming languages, dependencies, and architecture.

\subsection{Performance Evaluation}

Explain the benchmarking methodology and performance metrics used.

\section{Results}

\subsection{Performance Benchmarks}

Present performance comparisons with existing tools.

\subsection{Case Studies}

Provide real-world examples demonstrating OuterSpace's capabilities.

\section{Discussion}

Discuss the implications of the results, limitations, and future directions.

\section{Conclusion}

Summarize the key contributions and impact of OuterSpace.

\section*{Acknowledgments}

CNHC Grant.
Wigdahl RO1 Grant.

\section*{Data Availability}

OuterSpace is freely available at \url{https://github.com/DamLabResources/outerspace}.
All data and code used in this study are available upon request.

\section*{Author Contributions}

WND conceived the project.
All authors contributed to the design and implementation of OuterSpace.
All authors contributed to writing and reviewing the manuscript.

\section*{Competing Interests}

The authors declare no competing interests.

% bioRxiv bibliography style
\bibliographystyle{unsrt}
\bibliography{../bib/bibliography}

\end{document}
